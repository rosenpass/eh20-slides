\documentclass{rosenpass-beamer}

\usepackage[german]{babel}
\usepackage[autostyle]{csquotes}
\usepackage{emoji}
%\usepackage{dirtytalk}
\let\say\enquote

\usepackage{xurl}

\urlstyle{same}

\usepackage{textcomp}

\usetikzlibrary{positioning,decorations.pathreplacing,svg.path}

\definecolor{RPPink}{rgb}{274,4,132}
\definecolor{RPOrange}{rgb}{255, 166, 48}
\definecolor{RPAquamarine}{rgb}{255, 166, 48}
\definecolor{RPLightGray}{rgb}{160, 159, 164}
\definecolor{RPTurquoise}{rgb}{114, 161, 229}

\title{Rosenpass}
\subtitle{Update MRMCD 2023}
\author{kora, emil, ajuvo (@chaos.social)}
\institute{\url{https://rosenpass.eu}}


\conference{MRMCD 2023}
\date{2023-09-03}

\parskip\smallskipamount

\ExplSyntaxOn
\newcommand*{\sourcename}{Quelle}
\newcommand*{\sourcesep}{:~}
\newcommand*{\ImgSource}[2]{%
%\hbox_set:Nn \l_tmpa_box {#1}

\hbox_set:Nn \l_tmpa_box {#1}

\hbox_set:Nn \l_tmpa_box {
\dim_set:Nn \l_tmpa_dim {\box_ht:N \l_tmpa_box}
\hbox_unpack_drop:N \l_tmpa_box \rotatebox{90}{\parbox{\l_tmpa_dim}{\tiny\sourcename\sourcesep#2}}
}
\box_use:N \l_tmpa_box
}
\ExplSyntaxOff

% reduce itemize indent
\setlength{\leftmargini}{0pt}

\usepackage{biblatex}
\addbibresource{sources.bib}


%namepartpicturesetup
\ExplSyntaxOn
\int_new:N \l__ptxcd_namepart_int
\fp_new:N \l__ptxcd_namepos_fp
\def\namepartsep{1.1}
\dim_new:N \l__ptxcd_namepart_sep_dim
\dim_set:Nn \l__ptxcd_namepart_sep_dim  {3mm}

\newcommand*{\namepart}[2][0]{
	\int_set:Nn \l__ptxcd_namepart_int {\clist_count:n {#2}}
	\begin{scope}[xshift=#1]
	\fp_set:Nn \l__ptxcd_namepos_fp {\l__ptxcd_namepart_int / 2}
	\keyval_parse:nnn {\__ptxcd_namepart_item:nn {}}{ \__ptxcd_namepart_item:nn } {#2}
	\end{scope}
}

\newcommand*{\SingleNamePart}[4][0]{
		\node[rounded~corners,fill=rosenpass-lightblue] (#2) at (#1,-.7) {\ttfamily#3};
		\node[above] at (#2.north) {\footnotesize #4};
}

\cs_new:Nn \__ptxcd_namepart_item:nn {
	\fp_sub:Nn \l__ptxcd_namepos_fp {1}
	\node[rounded~corners,fill=rosenpass-lightblue] (#1) at (0,\fp_use:N \l__ptxcd_namepos_fp * \namepartsep) {\ttfamily#1};
	\node[above] at (#1.north) {\footnotesize #2};
}


\newcommand*{\namebraceleft}[2] {
	\draw[decorate]([xshift=-\l__ptxcd_namepart_sep_dim]#2.south~west)--([xshift=-\l__ptxcd_namepart_sep_dim]#1.north~west) ;
}

\newcommand*{\namebraceright}[2]{
	\draw[decorate]([xshift=\l__ptxcd_namepart_sep_dim]#1.north~east) --([xshift=\l__ptxcd_namepart_sep_dim]#2.south~east);
}
\ExplSyntaxOff

\graphicspath{{}{graphics/}}

\begin{document}

\maketitle


% Folie 2:
	\begin{frame}{Recap}
		\begin{itemize}
			\item Release Vortrag zum Easterhegg23
				\begin{itemize}
					\item Media.ccc.de > Rosenpass
				\end{itemize}
			
			\item Whitepaper
				\begin{itemize}
					\item Rosenpass.eu > whitepaper
				\end{itemize}
			
			\item @rosenpass@chaos.social
	
			\item Github > rosenpass
				\begin{itemize}
					\item https://github.com/rosenpass/rosenpass
				\end{itemize}
		\end{itemize}
	\end{frame}
		
		
% Folie 3:
	\begin{frame}{Was ist Rosenpass}
		\begin{itemize}
			\item Ein Open-Source Software-Projekt, das Post-Quantum-sichere Kryptographie zur Verfügung stellt
			\item Erste Implementierung: PQS für WireGuard
		\end{itemize}
	\end{frame}		


% Folie 4:
	\begin{frame}{Warum PQS}
		\begin{itemize}
			\item Rosenpass nutzt das McEliese-Kryptosystem
			\item Store now, decrypt later
			\item Der „Quantencomputer“
			\item Die praktische Relevanz
			\item Warum jetzt?
		\end{itemize}
	\end{frame}


% Folie 5:
	\begin{frame}{What next}
		\begin{itemize}
			\item WireGuard ist erst der Anfang
				\begin{itemize}
					\item Cloud
					\item Messenger
					\item Mail
					\item Netzwerke
					\item alles	
				\end{itemize}
		\end{itemize}
	\end{frame}	


% Folie 6:
	\begin{frame}{Nicht nur Software}
		\begin{itemize}
			\item (Wissenschafts-) Kommunikation
			\item Kryptographie = Mathematik und Vertrauen
			\item Propagation und Translation
		\end{itemize}
	\end{frame}	


% Folie 7:
	\begin{frame}{Was seitdem geschah}
		\begin{itemize}
			\item Team Evolution
			\item Finanzierung
			\item Kooperation mit Wirtschaft und Wissenschaft
			\item Institution und Organisation: Rosenpass e.V.
		\end{itemize}
	\end{frame}


% Folie 8:
	\begin{frame}{Was jetzt gerade geschieht}
		\begin{itemize}
			\item Formaler Beweis
			\item Distributionen
			\item Implementationen
			\item Neue Versionen
		\end{itemize}
	\end{frame}	


% Folie 9:
	\begin{frame}{Was jetzt gerade geschieht}
		\begin{itemize}
			\item Vernetzung in Organisation und Konsortium
			\item Finanzierung
			\item Evolution in der Kommunikation: Neue Medienformate
			\item Dialog mit Wissenschaft, Entwicklern, Dienstleistern, Anwendern
		\end{itemize}
	\end{frame}	


% Folie 10:
	\begin{frame}{Was bald geschieht}
		\begin{itemize}
			\item Releases
			\item Distributionen
			\item Neue Anwendungen
			\item Papers,Talks, Medien
		\end{itemize}
	
		\begin{itemize}
			\item \textbf{Spaß am Gerät und zusammen mit Euch !}
		\end{itemize}
	\end{frame}	


% Folie 11:
	\begin{frame}{Hoffentlich: Fragen!}
		\begin{itemize}
			\item rosenpass.eu
			\item github > rosenpass
			\item onboarden?
		\end{itemize}
	
		\begin{itemize}
			\item \textbf{Vielen Dank !}
			\begin{itemize}
				\item \textbf{kora   emil   ajuvo}
			\end{itemize}	
		\end{itemize}
	\end{frame}	

\end{document}
