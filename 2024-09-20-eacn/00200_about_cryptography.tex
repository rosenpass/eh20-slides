\interlude[0]{About cryptology}
\section{About cryptology}

\begin{frame}{To build real-world cryptography solutions}
  TODO
\end{frame}

\begin{frame}{Proofs of security are fundamental: Reduction proofs}
  \small
  Proof by reduction to a well-known mathematical problem \emph{assumed to be hard} or existing cryptographic construction \emph{assumed to be secure}.
  \vfill
  \textbf{If} an attack against my cryptosystem exists,
  \\ \textbf{then} this other cryptosystem can be attacked or this math problem can be solved.
  \vfill
  Proof ad-absurdum:
  \begin{itemize}
    \item Assume an attacker against the new cryptosystem exists
    \item Construct a solution to the underlying math problem using the assumed attacker
  \end{itemize}
\end{frame}

\begin{frame}{Proofs are fundamental: Using information theory}
  \small

  Showing that each \emph{plain text} is plausible for each \emph{ciphertext}.

  \vfill

  Cryptosystem should be formulated as a function:

  $$F : K \times D \to C$$

  \begin{description}
    \item[$K$] Key material; secret information held by the trusted parties
    \item[$D$] Protected information
    \item[$C$] Leaked information; any information known to the attacker after protocol execution
  \end{description}

  Now it needs to be shown, that for every value of the leaked information,
  every value of the protected information is equally plausible.


  $$\forall c : C, d_1 : D, d_2 : D; |\{ k \in K | F(k, d_1) = c \}| = |\{ k \in K | F(k, d_2) = c \}|$$
\end{frame}

\begin{frame}{Proofs are fundamental: Implementation security}
  \begin{columns}[t,fullwidth]

    \begin{column}{.40\textwidth}
      \begin{block}{Functional Correctness}
        Using formal methods from computer science that a cryptographic implementation is functionally equivalent to its specification.
      \end{block}
      
      \vfill

      \begin{block}{Efficiency}
        Using complexity theoretic analysis to ensure that the implementation can not be slowed down by an attacker.
      \end{block}
    \end{column}

    \begin{column}{.55\textwidth}
      \begin{block}{Implementation security}
        Ensuring cryptographic implementation fulfill various extra security properties.
        For example:
        \vfill
        \begin{itemize}
          \item Timing side-channel resistance (certain assembly operations are forbidden)
          \item Memory-safety (advanced programming languages such as Rust to avoid bugs such as buffer-overflows)
        \end{itemize}
      \end{block}
    \end{column}
  \end{columns}

\end{frame}

\begin{frame}{Practical security is essential}
  \small

  It is not enough to build a system that is secure in theory but vulnerable on real hardware.
  Some dangers include:

  \begin{itemize}
    \item Timing side-channels
    \item Power side-channels
    \item Hardware bugs in the CPU (Rowhammer, Spectre, or Meltdown)
    \item Lack of usability (Implementations that are easy to misuse)
  \end{itemize}
\end{frame}


\begin{frame}{Implementations and specifications must be open}
  % TODO(marei): Can we render this as full-size background image
  \includegraphics{graphics-repo/misc/kryptoparty.jpg}

  % TODO(karolin): Copy/paste kerkoffs principle
\end{frame}

\begin{frame}{Open-Source \& Open-Science are mandatory}
  \begin{columns}[fullwidth,T]
    \only<1>{
      \begin{column}{.6\textwidth}
        \includegraphics[width=.45\textwidth]{graphics-repo/comic/rosenpass-comic_01-obscurity1.jpg}
        \includegraphics[width=.45\textwidth]{graphics-repo/comic/rosenpass-comic_01-obscurity2.jpg}

        \includegraphics[width=.45\textwidth]{graphics-repo/comic/rosenpass-comic_01-obscurity3.jpg}
        \includegraphics[width=.45\textwidth]{graphics-repo/comic/rosenpass-comic_01-obscurity4.jpg}
      \end{column}

      \begin{column}{.4\textwidth}
        Keeping details about a security system secret creates mistrust and risks obscuring obvious security flaws
      \end{column}
    }

    \only<2>{
      \begin{column}{.4\textwidth}
        Cryptography is about creating trust; so peer review and open processes are a crucial part of the process.
      \end{column}

      \begin{column}{.6\textwidth}
        \includegraphics[width=.45\textwidth]{graphics-repo/comic/rosenpass-comic_02-disclosure1.jpg}
        \includegraphics[width=.45\textwidth]{graphics-repo/comic/rosenpass-comic_02-disclosure2.jpg}
      \end{column}
    }
  \end{columns}
\end{frame}

\begin{frame}{More than encryption}
  \small
  Key-exchanges are a subfield in cryptography, not the whole thing!

  \begin{description}
    \item[Multi-Party Computation] Arbitrary computation on encrypted data without cheating by consortium
    \item[Homomorphic Encryption] Arbitrary computation on asymmetrically encrypted data
    \item[Robust Combiners] Redundancy in cryptographic systems
    \item[Private Information Retrieval] Databases without leakage about user activity
    \item[Censorship circumvention]
  \end{description}
\end{frame}

\begin{frame}{To integrate QKD in cryptology}
  \small
  \begin{itemize}
    \item Integrate with community of cryptography researchers
    \item Adopt comprehensive approach of cryptology
    \item Adopt open-source/open-science approach
    \item Define security properties in cryptographic terms to be comparable
    \item \textbf{Use QKD within cryptographic systems}, not as an alternative to!
  \end{itemize}

  The field we are looking at when building cryptographic systems including QKD is called:

  \centering\large Secure Channels
\end{frame}
